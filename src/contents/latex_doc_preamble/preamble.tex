%--------------------------------------------------------------------------------------------------
% Preamble: Packages Used and Format Settings
%--------------------------------------------------------------------------------------------------

%--------------------------------------------------------------------------------------------------
% General Packages Used
%--------------------------------------------------------------------------------------------------
\usepackage{a4} % Set paper size to A4
\usepackage{ragged2e} % Better alignment options
\usepackage{amsmath,amssymb} % Enhanced mathematical typesetting
\usepackage{caption} % Customizing captions for figures and tables
\usepackage[export]{adjustbox} % Adjusting positioning and scaling of boxes (e.g., figures)
\usepackage{stackengine} % Stack items vertically or horizontally
\usepackage{graphicx} % Enhanced support for graphics
\usepackage{caption} % Customizing captions for figures and tables (continued)
\captionsetup{justification=centering, singlelinecheck=false} % Centering figure and table captions
\usepackage{amsfonts} % Additional mathematical fonts
\usepackage{bm} % Bold mathematical symbols
\usepackage{multirow} % Multi-row cells in tables
\usepackage{float} % Improved float placement options
\usepackage{rotating} % Rotating objects (e.g., figures, tables)
\usepackage[version=4]{mhchem} % Typesetting chemical formulas
\usepackage{physics} % Simplifies typesetting of common mathematical physics notation
\usepackage{blindtext} % Generates dummy text for testing purposes
\usepackage{braket} % Typesetting bra-ket notation used in quantum mechanics
\usepackage{enumerate} % Customizing enumeration lists
\usepackage{color} % Adding color support
\usepackage{tikz} % Comprehensive support for creating graphics programmatically
\usepackage{pgfplots} % High-quality plots and diagrams
\renewcommand{\familydefault}{\rmdefault} % Switch to default (Roman) font
% --- If You want to use Helvetica Font, uncomment the following 3 lines:
% \usepackage{helvet} % Uncomment to use Helvetica font
% \renewcommand{\familydefault}{\sfdefault} % Set default font family to sans-serif (Helvetica)
% \usepackage[helvet]{sfmath} % Use Helvetica font for math symbols
\usepackage[ruled,vlined]{algorithm2e} % Typesetting algorithms
\usepackage{fancybox} % Produces simple frames and boxes
\usepackage{verbatim} % Supports multiline comments
\usepackage[version=4]{mhchem} % Typesetting chemical formulas (continued)
\pgfplotsset{compat=1.16} % Set compatibility level for pgfplots
\usepackage{setspace} % Customizing line spacing
\onehalfspacing % Set line spacing to 1.5
\usepackage{fancyhdr} % Customizing page headers and footers
\pagestyle{plain} % Set page style to plain (page number centered at the bottom)
\usepackage[sorting=none]{biblatex} % Enhanced bibliography management
\bibliography{src/contents/backmatter/references/references} % Specify bibliography file
\usepackage{sectsty} % Package for customizing section titles
\usepackage{acronym} % Package for managing acronyms
% \usepackage{titletoc} % For customizing TOC formatting
% % Redefine TOC font to sffamily
% \titlecontents{part}[0em]{\vspace{1em}\sffamily\bfseries\large}{\thecontentslabel\hspace{1em}}{}{\titlerule*[0.5pc]{}\contentspage}
% \titlecontents{chapter}[0em]{\vspace{1em}\sffamily\bfseries}{\thecontentslabel\hspace{1em}}{}{\titlerule*[0.5pc]{}\contentspage}
% \titlecontents{section}[1em]{\vspace{0.5em}\sffamily}{\thecontentslabel\hspace{1em}}{}{\titlerule*[0.5pc]{}\contentspage}
% \titlecontents{subsection}[2em]{\vspace{0.5em}\sffamily}{\thecontentslabel\hspace{1em}}{}{\titlerule*[0.5pc]{.}\contentspage}
% \titlecontents{subsubsection}[3em]{\vspace{0.5em}\sffamily}{\thecontentslabel\hspace{1em}}{}{\titlerule*[0.5pc]{.}\contentspage}
\usepackage{booktabs} % For professional-looking tables
\usepackage{makeidx} % Package for creating index
\makeindex % Command to create the index

%--------------------------------------------------------------------------------------------------
% Code Block Formatting and Syntax Highlighting
%--------------------------------------------------------------------------------------------------
\usepackage{listings} % Package for typesetting code
\usepackage{xcolor}   % Package for defining colors
% Redefine the heading of the list of listings
\renewcommand{\lstlistlistingname}{List of Code Snippets}

% --------------Themes------------------

% Themes were generated using chatgpt

% Define colors for a light themed C++ code
\definecolor{codegreen}{rgb}{0,0.6,0}        % Define color for comments
\definecolor{codegray}{rgb}{0.5,0.5,0.5}      % Define color for line numbers
\definecolor{codepurple}{rgb}{0.58,0,0.82}    % Define color for keywords
\definecolor{backcolour}{rgb}{0.95,0.95,0.92} % Define background color

% Define listing style for light theme
\lstdefinestyle{cpplight}{
    language=C++,              % Set code language
    backgroundcolor=\color{backcolour},   % Set background color
    commentstyle=\color{codegreen},       % Set comment style
    keywordstyle=\color{magenta},         % Set keyword style
    numberstyle=\tiny\color{codegray},    % Set line number style
    stringstyle=\color{codepurple},       % Set string style
    basicstyle=\footnotesize\ttfamily,    % Set basic code style
    breakatwhitespace=false,              % Don't break lines at whitespace
    breaklines=true,                      % Break lines if too long
    captionpos=b,                         % Set caption position
    keepspaces=true,                      % Keep spaces
    numbers=left,                         % Show line numbers on left
    numbersep=5pt,                        % Set space between line numbers and code
    showspaces=false,                     % Don't show spaces
    showstringspaces=false,               % Don't show spaces in strings
    showtabs=false,                       % Don't show tabs
    tabsize=4                             % Set tab size
}

% Dracula colors (RGB)
\definecolor{dracula-bg}{RGB}{40, 42, 54}         % Background color
\definecolor{dracula-fg}{RGB}{248, 248, 242}      % Foreground color
\definecolor{dracula-comment}{RGB}{98, 114, 164}  % Comment color
\definecolor{dracula-cyan}{RGB}{80, 250, 123}     % Cyan color
\definecolor{dracula-green}{RGB}{80, 250, 123}    % Green color
\definecolor{dracula-orange}{RGB}{255, 184, 108}  % Orange color
\definecolor{dracula-pink}{RGB}{255, 121, 198}    % Pink color
\definecolor{dracula-purple}{RGB}{189, 147, 249}  % Purple color
\definecolor{dracula-red}{RGB}{255, 85, 85}       % Red color
\definecolor{dracula-yellow}{RGB}{241, 250, 140}  % Yellow color

% Dracula style for Python
\lstdefinestyle{python-dracula}{
    language=Python,                               % Set code language
    backgroundcolor=\color{dracula-bg},            % Background color
    basicstyle=\color{dracula-fg}\ttfamily,        % Basic code style
    commentstyle=\color{dracula-comment},          % Comment style
    keywordstyle=\color{dracula-pink},             % Keyword style
    stringstyle=\color{dracula-yellow},            % String style
    numbers=left,                                  % Show line numbers on left
    numberstyle=\color{black},                     % Line number style
    numbersep=10pt,                                % Space between line numbers and code
    breaklines=true,                               % Break lines if too long
    keepspaces=true,                               % keeps spaces in text, useful for keeping indentation of code (possibly needs columns=flexible)
    showspaces=false,                              % Don't show spaces
    showstringspaces=false,                        % Don't show spaces in strings
    frame=single,                                  % Add frame around code
    tabsize=4,                                     % Set tab size
    breakatwhitespace=true                         % Break lines at whitespace
    rulecolor=\color{dracula-bg},                  % Set rule color to background color
    aboveskip=2em,                                 % Set space above code block
    belowskip=2em                                  % Set space below code block
}

% Dracula style for C++
\lstdefinestyle{cpp-dracula}{
    language=C++,                                  % Set code language
    backgroundcolor=\color{dracula-bg},            % Background color
    basicstyle=\color{dracula-fg}\ttfamily,        % Basic code style
    commentstyle=\color{dracula-comment},          % Comment style
    keywordstyle=\color{dracula-pink},             % Keyword style
    stringstyle=\color{dracula-yellow},            % String style
    numbers=left,                                  % Show line numbers on left
    numberstyle=\color{black},                     % Line number style
    numbersep=10pt,                                % Space between line numbers and code
    breaklines=true,                               % Break lines if too long
    keepspaces=true,                               % keeps spaces in text, useful for keeping indentation of code (possibly needs columns=flexible)
    showspaces=false,                              % Don't show spaces
    showstringspaces=false,                        % Don't show spaces in strings
    frame=single,                                  % Add frame around code
    tabsize=4,                                     % Set tab size
    breakatwhitespace=true                         % Break lines at whitespace
    rulecolor=\color{dracula-bg},                  % Set rule color to background color
    aboveskip=2em,                                 % Set space above code block
    belowskip=2em                                  % Set space below code block
}

% Light theme for Python (color-blind friendly)
\lstdefinestyle{python-github-light}{
    language=Python,
    backgroundcolor=\color{white},
    basicstyle=\small\ttfamily\color{black},
    keywordstyle=\color{blue},
    commentstyle=\color{green!60!black},
    stringstyle=\color{red!70!black},
    numbers=left,
    numberstyle=\color{gray},
    numbersep=5pt,
    showspaces=false,
    showstringspaces=false,
    tabsize=4,
    breaklines=true,
    frame=single,
    rulecolor=\color{black},
    captionpos=t, % Caption at the top
    aboveskip=2em,
    belowskip=2em
}

% Light theme for C++ (color-blind friendly)
\lstdefinestyle{cpp-github-light}{
    language=C++,
    backgroundcolor=\color{white},
    basicstyle=\small\ttfamily\color{black},
    keywordstyle=\color{blue},
    commentstyle=\color{green!60!black},
    stringstyle=\color{red!70!black},
    numbers=left,
    numberstyle=\color{gray},
    numbersep=5pt,
    showspaces=false,
    showstringspaces=false,
    tabsize=4,
    breaklines=true,
    frame=single,
    rulecolor=\color{black},
    captionpos=t, % Caption at the top
    aboveskip=2em,
    belowskip=2em
}

% Dark theme for Python (color-blind friendly)
\lstdefinestyle{python-github-dark}{
    language=Python,
    backgroundcolor=\color{black},
    basicstyle=\small\ttfamily\color{white},
    keywordstyle=\color{cyan},
    commentstyle=\color{green!60!black},
    stringstyle=\color{orange},
    numbers=left,
    numberstyle=\color{gray},
    numbersep=5pt,
    showspaces=false,
    showstringspaces=false,
    tabsize=4,
    breaklines=true,
    frame=single,
    rulecolor=\color{white},
    captionpos=t, % Caption at the top
    aboveskip=2em,
    belowskip=2em
}

% Dark theme for C++ (color-blind friendly)
\lstdefinestyle{cpp-github-dark}{
    language=C++,
    backgroundcolor=\color{black},
    basicstyle=\small\ttfamily\color{white},
    keywordstyle=\color{cyan},
    commentstyle=\color{green!60!black},
    stringstyle=\color{orange},
    numbers=left,
    numberstyle=\color{gray},
    numbersep=5pt,
    showspaces=false,
    showstringspaces=false,
    tabsize=4,
    breaklines=true,
    frame=single,
    rulecolor=\color{white},
    captionpos=t, % Caption at the top
    aboveskip=2em,
    belowskip=2em
}

% --------------Themes------------------

%--------------------------------------------------------------------------------------------------
% Multifile Package
%--------------------------------------------------------------------------------------------------
\usepackage{standalone} % Handling standalone files
\usepackage{import} % Improved handling of import commands
\usepackage{tocbibind} % Include table of contents, list of figures, etc., in table of contents
% NOTE: use hyperref as the last package
\usepackage[hidelinks]{hyperref} % Enhancements for hyperlinks and cross-references
% \hypersetup{
%   colorlinks   = true, %Colours links instead of ugly boxes
%   urlcolor     = blue, %Colour for external hyperlinks
%   linkcolor    = blue, %Colour of internal links
%   citecolor   = red %Colour of citations
% }
\usepackage{bookmark}
% Custom command for referencing figures with custom text
\newcommand{\figref}[1]{\hyperref[#1]{Figure~\ref*{#1}}}